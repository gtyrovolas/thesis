%!TEX output_directory = .aux
%!TEX copy_output_on_build(true)

\documentclass[11pt,a4paper, titlepage]{report}
\usepackage[a4paper, total={6.5in, 8in}]{geometry}
\usepackage[utf8]{inputenc}
\usepackage{amsfonts}
\usepackage{amssymb}
\usepackage{amsmath}
\usepackage{mathtools}
\usepackage{amsthm}

\title{An Analysis of smart voting in liquid democracy}
\author{Giannis Tyrovolas}
\date{\today}
\newtheorem{theorem}{Theorem}[section]
\newtheorem{prop}[theorem]{Proposition}
\newtheorem*{remark}{Remark}
\DeclarePairedDelimiter\abs{\lvert}{\rvert}
\DeclarePairedDelimiter\norm{\lVert}{\rVert}

\theoremstyle{definition}
\newtheorem{definition}[theorem]{Definition}
\newtheorem{example}[theorem]{Example}
\newtheorem{corollary}[theorem]{Corollary}
\newtheorem{lemma}[theorem]{Lemma}
\newtheorem{proposition}[theorem]{Proposition}
\newtheorem*{idea}{Idea}

\begin{document}

\maketitle

\tableofcontents

\chapter{Preliminaries}

\section{Notation}


An election consists of a finite set of voters, voting on a finite set of issues. For each issue there is a finite set of alternatives. A special alternative is the abstention represented by $*$. Formally:

\begin{definition}[Election]
    An \emph{election} consists of a tuple $\langle N, I, D\rangle $ where $N = \{1,..., n\} $ is a finite non-empty set. The finite non-empty set $I$ holds represents the issues of the election. For each $i \in I$, $D(i)$ is a finite non-empty set.
\end{definition}

For single-issue elections we omit the reference to $I$ and abbreviate $D(i)$ to $D$.

The model we will consider allows each voter to submit a smart ballot. A smart ballot is a preference list of smart votes. Each smart vote is a function with domain the other voters. A special requirement is that the final preference in the preference list is a direct vote on an alternative in $D$. Formally:

\begin{definition}[Smart Ballots]
    A smart ballot of an agent $a$ on issue $i \in I$ is an ordering $( (S^1, F^1) > \ldots > (S^k, F^k) > d)$ where $k \geq 0$. Each $S^h$ for $h \leq k$ is a subset of $N$ and $F^h \colon D(i)^{S^h} \longrightarrow D(i)$ is a resolute aggregation function. We also have that $d \in D(i)$.   
\end{definition}

Further when relevant we will consider $F^{k+1}$ to be the constant function with output $d$. Now, in most cases the sets $S^h$ are implicit and we will drop any mention to them. That is supported by the fact that we will treat two functions $F, G$ as identical if they are extensionally equal. This is formalised by the following definition:

\begin{definition}[Valid Smart Ballot]
    A valid smart ballot of an agent $a$ is a smart ballot $B_a$ such that for all $ 1 \leq s < t \leq k + 1$ $F^s$ is not equivalent to $F^t$. Additionally $a \notin S_t$.
\end{definition}

\chapter{Results}

\iffalse
Plan for this section:
1) Cast participation is achieved only for monotone functions. So, let's focus on monotone functions
2) Prove NP-hardness for MinMax, MinSum using only binary or and binary and.
3) Prove that no constant approximator to MinMax or MinSum exists.
4) Result on whether we are able to even determine whether a vote is achievable.
5) Generalise on why the proof works for all non-delegative, fair and monotone functions. 
\fi


\chapter{Proposals}

\end{document}

